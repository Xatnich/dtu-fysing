% \iffalse meta-comment
%
% Copyright (C) 2025 Max Rufus Schepelern Karrebæk
%
% This file may be distributed and/or modified under the
% conditions of the LaTeX Project Public License, either
% version 1.3 of this license or (at your option) any later
% version. The latest version of this license is in:
%
% http://www.latex-project.org/lppl.txt
%
% and version 1.3c or later is part of all distributions of
% LaTeX version 2008-05-04 or later.
%
% \fi
%
% \iffalse
%<package>\NeedsTeXFormat{LaTeX2e}
%<package>\RequirePackage{amssymb}
%<package>\ProvidesPackage{dtufysing}[2025-06-09 v0.0.2 by Max Karrebæk for use in notes and written material as part of studies associated with BSc in engineering physics at DTU.]
%
%<*driver>
\documentclass{ltxdoc}
\usepackage{dtufysing}
\usepackage{array}
\usepackage{amsmath}
\EnableCrossrefs
\CodelineIndex
\RecordChanges
\begin{document}
\DocInput{dtufysing.dtx}
\end{document}
%</driver>
% \fi
%
%\changes{v0.0.1}{2025-06-08}{Initial non-working version}
%\changes{v0.0.2}{2025-06-09}{Initial working version}
%\GetFileInfo{dtufysing.sty}
%\filedate
%\fileversion
%\fileinfo
%
% \title{The \textsf{DTUFysIng} package\thanks{This document corresponds to \textsf{DTUFysIng}~\fileversion, dated~\filedate.}}
% \author{Max Rufus Schepelern Karrebæk \\ \texttt{s245173@dtu.dk}, \texttt{max@fysikraadet.pf.dk}}
%
% \maketitle
%
% \section{Usage and examples}
% \subsection{Easing adherence to ISO standards}
% ISO requires that dimensionless constants (numbers) such as $\sqrt{-1}$ are typeset in \textrm{roman} font whereas physical dimensions are set in an \textit{italiziced} font. Not surprisingly, being strict turns out to be useful eg. as part of an analysis of a mechanical system, you might have introduced the distance $d$. Later on, it is necessary to multiply the time derivate of some dimension $x$ by this distance. This can be written either incorrectly or the correct way:
% \begin{align}
%	d\frac{dx}{dt}&\quad\text{incorrect and confusing}\\
%    d\dif[x]{}{t}&\quad\text{correct and understandable}
% \end{align}
% We therefore define the following macros.
%
% \DescribeMacro{\im}
% \DescribeMacro{\e}
% \DescribeMacro{\dd}
% These macros should only be used in math mode. Their use can be seen in table~\ref{tab:rmnumbers}. 
% \begin{table}[h!]
% \centering
% \begin{tabular}{l>{\(}r<{\)}}
% |a + b\im| & a + b\im \\
% |\e^{-x^2}=\exp{(-x^2)}| 	& \e^{-x^2}=\exp{(-x^2)} \\
% |\dd x| 				& \dd x
% \end{tabular}
% \caption{Roman numbers.}
% \label{tab:rmnumbers}
% \end{table}
%  Electrical engineers should redefine the macro |\im| by adding |\def\im{\mathrm{j}}| to their preamble---maybe this will be a package option in the future.
%
% \subsection{Calculus}
% \DescribeMacro{\diff}
% \DescribeMacro{\pdiff}
% Lists of differentials can be written as |$\diff{x,y,z}$| yielding $\diff{x,y,z}$. This is very useful when doing vector calculus!
% Consider the following integral.
% \begin{equation}
% \iiint x+y+z\diff{x,y,z}
% \end{equation}
% This can be written in the compact and readable expression |\iiint x+y+z\diff{x,y,z}| as opposed to writing the integrands explicitly |\,\dd x\,\dd y\,\dd z| or worse, by using |\mathrm{d}|. A corresponding command |\pdiff{x,y,z}| is also supplied: $\pdiff{x,y,z}$. The interproduct spacing can be modified using an optional argument |\pdiff[]{\alpha,\beta}|: $\pdiff[]{\alpha,\beta}$
%
% \DescribeMacro{\dif}
% \DescribeMacro{\pdif}
% \textsf{DTUFysIng} makes it very easy to write the differential quotient. This can be achieved using either |\dif| og |\pdif|:
% \begin{equation}
% \dif[x]{2}{t} + \omega^2\dif[x]{}{t}=0
% \end{equation}
% Simply, by writing |\dif[x]{2}{t} + \omega^2\dif[x]{}{t}=0|. More complicated expressions are also allowed
% \begin{equation}
% \pdif{3}{x^2,y}f(x,z)=0
% \end{equation}
% |\pdif{3}{x^2,y}f(x,z)=0|
%
% \subsubsection{Vectors}
% \DescribeMacro{\vect}
% Vectors are defined using |\vect|. Maybe the macro should just redefine the plain command |\vec|. Vectors are set as boldsymbol: |\vect{f}(\vect{x})| yields $\vect{f}(\vect{x})$.
%
% \DescribeMacro{\dive}
% \DescribeMacro{\rot}
% \DescribeMacro{\grad}
% The divergence, rotation and gradient of a vector field can be written as
% \begin{equation}
% \rot{(\rot{\vect{c}})} = \grad{(\dive{\vect{c}})} - \nabla^2\vect{c}
% \end{equation}
% |\rot{(\rot{\vect{c}})} = \grad{(\dive{\vect{c}})} - \nabla^2\vect{c}|.
% Divergence is |\dive| and not |\div| as this macro is already defined to be the (useless) symbol $\div$.
%
% \MaybeStop{\PrintIndex\par\PrintChanges}
%
% \section{Implementation}
% \begin{macro}{\im}
% \begin{macro}{\e}
% \begin{macro}{\dd}
% These macros should probably be renamed to
% |\imaginary|, |\euler| and |\differentiald|, for compatibility reasons. 
% The user would then be advised to define shorthand macros: 
% |\newcommand*\im\imaginary| etc. Maybe for v0.0.3.
%    \begin{macrocode}
\def\im{\mathrm{i}}%
\def\e{\mathrm{e}}%
\def\dd{\mathrm{d}}%
%    \end{macrocode}
%\end{macro}
%\end{macro}
%\end{macro}
%
%\begin{macro}{\ddspace}
%|\ddspace| is an internal macro used in differentials. Maybe it should be defined as a length and not just a macro\ldots Another project for v0.0.3!
%    \begin{macrocode}
\let\ddspace=\thinspace%
%    \end{macrocode}
%\end{macro}
%
%\begin{macro}{\diff}
%\begin{macro}{\pdiff}
% They are both wrappers for the internal macro |\@iff|. They share the same syntax, |\diff|\oarg{spacing}\marg{comma-separated list}.
%    \begin{macrocode}
\def\@diff#1#2#3{\@for\@iff:={#1}\do{#2#3\@iff}}%
\newcommand*{\diff}[2][\ddspace]{\@diff{#2}{#1}{\dd}}%
\newcommand*{\pdiff}[2][\ddspace]{\@diff{#2}{#1}{\partial}}%
%    \end{macrocode}
%\end{macro}
%\end{macro}
%
%This counter will be useful. Alongside some internal macros. It is somewhat arbitrarily set to zero.
%    \begin{macrocode}
\newcount\len@diff%
\len@diff=\z@%
%    \end{macrocode}
% \begin{macro}{\@lenlist}
% Internal macro that measures whether a cs-list has one or more entries. If there is one entry the counter |@lenlist| is set to |1| and |2| otherwise. Maybe it should check whether there are entries.
%    \begin{macrocode}
\def\@lenlist#1,#2\@nil{%
  \def\@tempa{#2}
  \ifx\@tempa\@empty%
    \len@diff=\@ne
  \else
    \len@diff=\tw@
  \fi%
}%
%    \end{macrocode}
% \end{macro}
%
%\begin{macro}{\dif}
%\begin{macro}{\pdif}
%First, the internal command for differential quotients. This is purely internal.
%    \begin{macrocode}
\def\@dif#1#2#3#4{%
  \@lenlist#3,\@nil
  \ifnum\len@diff=1
    \def\@tempa{#2}
    \ifx\@tempa\@empty
      \frac{#4#1}{#4#3}
    \else
      \frac{#4^{#2}#1}{#4#3^{#2}}
    \fi
  \else
    \frac{#4^{#2}#1}{\@diff{#3}{}{#4}}
  \fi%
}%
%    \end{macrocode}
% |\dif| and |\pdif| should be used. They feature the same syntax, |\dif|\oarg{function}\marg{degree}\marg{denominator, list}. The argument \marg{degree} can be left empty which is normal for first degree differentials.
%    \begin{macrocode}
\newcommand*\dif[3][]{\@dif{#1}{#2}{#3}{\dd}}%
\newcommand*\pdif[3][]{\@dif{#1}{#2}{#3}{\partial}}%
%    \end{macrocode}
%\end{macro}
%\end{macro}
%vector notation
%    \begin{macrocode}
\def\vect#1{\boldsymbol{#1}}%
%    \end{macrocode}
%operators
%    \begin{macrocode}
\newcommand*\rot[1]{\vect{\nabla}\times #1}%
\newcommand*\dive[1]{\vect{\nabla}\cdot #1}%
\newcommand*\grad[1]{\vect{\nabla} #1}%
\newcommand*\del{\mathrm{\Delta}}%
%    \end{macrocode}
% \Finale
\endinput