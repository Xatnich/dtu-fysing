% \iffalse meta-comment
%
% Copyright (C) 2025 Max Rufus Schepelern Karrebæk
%
% This file may be distributed and/or modified under the
% conditions of the LaTeX Project Public License, either
% version 1.3 of this license or (at your option) any later
% version. The latest version of this license is in:
%
% http://www.latex-project.org/lppl.txt
%
% and version 1.3c or later is part of all distributions of
% LaTeX version 2008-05-04 or later.
%
% \fi
%
% \iffalse
%<package>\NeedsTeXFormat{LaTeX2e}
%<package>\RequirePackage{amssymb}
%<package>\ProvidesPackage{dtufysing}[2025-06-08 v0.0.1 by Max Karrebæk for use in notes and written material as part of studies associated with BSc in engineering physics at DTU.]
%
%<*driver>
\documentclass{ltxdoc}
\usepackage{dtufysing}
\EnableCrossrefs
\CodelineIndex
\RecordChanges
\begin{document}
\DocInput{dtufysing.dtx}
\end{document}
%</driver>
% \fi
%
%\changes{v0.0.1}{2025-06-08}{Initial non-working version}
%\GetFileInfo{dtufysing.sty}
%\filedate
%\fileversion
%\fileinfo
%
% \title{The \textsf{DTUFysIng} packagepackage\thanks{This document corresponds to \textsf{DTUFysIng}~\fileversion, dated~\filedate.}}
% \author{Max Rufus Schepelern Karrebæk \\ \texttt{s245173@dtu.dk, max@fysikraadet.pf.dk}}
%
% \maketitle
% \section{Usage and examples}
% \subsection{Easing adherence to ISO standards}
% ISO requires that dimensionless constants (numbers) such as $\sqrt{-1}$ are typeset in \textrm{roman} font whereas physical dimensions are set in an \textit{italiziced} font. We therefore define the following macros:
%
% \DescribeMacro{\im}
% \DescribeMacro{\e}
% \DescribeMacro{\dd}
% The macros should only be used in math mode: |$\im$| yields $\im$. 
%
%\subsection{Vector calculus}
% Lists of differentials can be written as $\diff{x,y,z}$. Very useful when doing vector calculus!
%Sometimes, we are not interested in index notation. 
%
% \MaybeStop{\PrintIndex}
%
% \section{Implementation}
%
%\begin{macro}{\im}
%\begin{macro}{\e}
%\begin{macro}{\dd}
%These macros should probably be renamed to |\imaginary|, |\euler| and |\differentiald|, for compatibility reasons. 
%The user would then be advised to define shorthand macros: |\newcommand*\im\imaginary| etc. Maybe for v0.0.2.
%\begin{macrocode}
\def\im{\mathrm{i}}%
\def\e{\mathrm{e}}%
\def\dd{\mathrm{d}}%
%\end{macrocode}
%\end{macro}
%\end{macro}
%\end{macro}
%
%\begin{macro}{\ddspace}
%|\ddspace| is an internal macro used in differentials. Maybe it should be defined as a length and not just a macro\ldots Another project for v0.0.2!
%\begin{macrocode}
\let\ddspace=\thinspace
%\end{macrocode}
%\end{macro}
%
%\begin{macro}{\diff}
%\begin{macro}{\pdiff}
% They are both wrappers for the internal macro |\@iff|. They share the same syntax, |\diff\oarg{spacing}\marg{list}|.
%\begin{macrocode}
\def\@diff#1#2#3{\@for\@iff:={#1}\do{#2#3\@iff}}
\newcommand*{\diff}[2][\ddspace]{\@diff{#2}{#1}{\dd}}
\newcommand*{\pdiff}[2][\ddspace]{\@diff{#2}{#1}{\partial}}
%\end{macrocode}
%\end{macro}
%\end{macro}
%
%This counter will be useful. Alongside some internal macros.
%\begin{macrocode}
\newcount\len@diff
\def\@ifempty#1{\def\@tempa{#1} \ifx\@tempa\@empty }
%\end{macrocode}
%\begin{macrocode}
\def\@lenlist#1,#2\@nil{%
  \@ifempty{#2}
    \len@diff=\@ne
  \else
    \len@diff=\tw@
  \fi
}
%\end{macrocode}
%
%\begin{macro}{\dif}
%\begin{macro}{\pdif}
%First, the internal command for differential quotients.
%\begin{macrocode}
\def\@dif#1#2#3#4{%
    \@lenlist#3,\@nil
    \ifnum\len@diff=1
        \def\@tempa{#2}
        \ifx\@tempa\@empty
        % \@ifempty{#2}
            \frac{#4#1}{#4#3}
        \else
            \frac{#4^{#2}#1}{#4#3^{#2}}
        \fi
    \else
        \frac{#4^{#2}#1}{\@diff{#3}{}{#4}}
    \fi
}
%\end{macrocode}
%These should be used. They feature the same syntax, |\dif|\oarg{function}\marg{degree}\marg{denominator, list}. The argument \marg{degree} can be left empty which is normal for first degree differentials.
%\begin{macrocode}
\newcommand*\dif[3][]{%
\@dif{#1}{#2}{#3}{\dd}
}
\newcommand*\pdif[3][]{%
\@dif{#1}{#2}{#3}{\partial}
}
%\end{macrocode}
%\end{macro}
%\end{macro}
%vector notation
%\begin{macrocode}
\def\vect#1{\boldsymbol{#1}}
%\end{macrocode}
%operators
%\begin{macrocode}
\newcommand*\rot[1]{\vect{\nabla}\times #1}
\newcommand*\dive[1]{\vect{\nabla}\cdot #1}
\newcommand*\grad[1]{\vect{\nabla} #1}
\newcommand*\del{\mathrm{\Delta}}
%\end{macrocode}
% \Finale
\endinput