\NeedsTeXFormat{LaTeX2e}
\ProvidesPackage{dtufysing}[2025/06/02 v0.0.1 by Max Karrebæk for use in notes and written material as part of studies associated with BSc in engineering physics at DTU]
\RequirePackage{amssymb}

%roman stuff
\def\im{\mathrm{i}}
\def\e{\mathrm{e}}
\def\dd{\mathrm{d}}

%%%%%%%%%%%%%%%%%%%%%
%calculus
%%%%%%%%%%%%%%%%%%%%%
\let\ddspace=\thinspace

%general command for lists of differentials
\def\@diff#1#2#3{\@for\@iff:={#1}\do{#2#3\@iff}}

\newcommand*{\diff}[2][\ddspace]{\@diff{#2}{#1}{\dd}}

\newcommand*{\pdiff}[2][\ddspace]{\@diff{#2}{#1}{\partial}}


\def\@ifempty#1{\def\@tempa{#1} \ifx\@tempa\@empty }

\newcount\len@diff

\def\@lenlist#1,#2\@nil{%
  %\def\@tempa{#2}%
  %\ifx\@tempa\@empty
  \@ifempty{#2}
    \len@diff=\@ne
  \else
    \len@diff=\tw@
  \fi
}
%general command for differential quotients
\def\@dif#1#2#3#4{%
    \@lenlist#3,\@nil
    \ifnum\len@diff=1
        \def\@tempa{#2}
        \ifx\@tempa\@empty
        % \@ifempty{#2}
            \frac{#4#1}{#4#3}
        \else
            \frac{#4^{#2}#1}{#4#3^{#2}}
        \fi
    \else
        \frac{#4^{#2}#1}{\@diff{#3}{}{#4}}
    \fi
}

\newcommand*\dif[3][]{%
\@dif{#1}{#2}{#3}{\dd}
}
\newcommand*\pdif[3][]{%
\@dif{#1}{#2}{#3}{\partial}
}

%%%%%%%%%%%%%%%%%%%
%vector notation
\def\vect#1{\boldsymbol{#1}}

%operators
\newcommand*\rot[1]{\vect{\nabla}\times #1}
\newcommand*\dive[1]{\vect{\nabla}\cdot #1}
\newcommand*\grad[1]{\vect{\nabla} #1}

\newcommand*{\del}{\mathrm{\Delta}}